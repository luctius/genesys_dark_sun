/FloatBarrier
\subsection{Magical and Psionics Gear}
\begin{multicols}{2}

\begin{table}[H]
\begin{GenesysTable}{Magic and Psionic Gear}{magic-and-psionic-gear}{ =X +l +l +l}
Name                                         & Encumbrance & Price      & Rarity \\
\nameref{itmmgc:esperweed}                   & 0           & (R) 2000   & 10 \\
\nameref{itmmgc:spice}                       & 0           & (R) 1000   & 7 \\
\nameref{itmmgc:healingfruit}                & 0           & 30         & 5 \\
\nameref{itmmgc:meditation-crystal}          & 1           & 4000       & 10 \\
\nameref{itmmgc:spellcomponentpouch}         & 1           & 50         & (I) 3 \\
\nameref{itmmgc:talisman-of-the-gaj}         & 1           & (R) 8000   & 10 \\
\end{GenesysTable}
\end{table}

\subsubsection{Esperweed} \label{itmmgc:esperweed}
Esperweed is a plant that grows in the few remaining tropical
areas of Athas, as well as on some of the mudflats surrounding
the Sea of Silt. A fairly rare plant by nature, esperweed is sought
after by many for its psionic boosting powers.
Natives of Athas have discovered that, when eaten, the roots of
esperweed can boost psionic powers.
For the next Psionic check, the psionic can add \dark\dark to their check,
but consuming it costs 1 strain.

\subsubsection{Yaladai Spice} \label{itmmgc:spice}
This high-end spice is relatively hard to come by. Extremely
addictive with long term use, it is especially useful to psionic
users looking to improve their abilities temporarily.

Effects last for 1 encounter.

Whenever the character suffers strain, suffer 3 less (to
minimum of 1).

If a psionic, add an additional force die to psionic power checks,
but any \dark used come at double strain, which do not get reduced
by this spice.

\subsubsection{Healing Fruit} \label{itmmgc:healingfruit}
Healing Fruits are fruit grown from magical infused fruit trees.
While using any fruit is possible, often pears are most common.
While they are great for infusing a living body with healing,
their effects do diminish quite fast. The first healing fruit
eaten on a day grants 5 healing, the second eaten the same day
4 healing and so forth.

\subsubsection{Meditation Crystal} \label{itmmgc:meditation-crystal}
These seoft light emiting crystals are found in the deep caverns of
the athasian desert and are both rare and difficult to mine. They
are usually worn around the fore-head, either within a headband or
on a delicate chain.

A psionic who has a meditation focus increases his strain threshold by 2.

\subsubsection{Spell Component Pouch}
\label{itmmgc:spellcomponentpouch}
A spell component pouch contains the spell components neccesary to cast spells, the
come with the pouch and are relatively common. There are also components which can
enhance or simplify the casting. Must be sought out by the spell caster and are either
expensive or difficult to gather. Only one component can be used per casting, and a
\despair can signal that the component has been exhausted for the scene. Unlike Primal
Components, Arcane Components can be damaged (but not repaired), or destroyed. See
\cref{sec:repairing-gear} for information regarding damage levels.

\begin{table*}[!htb]
\begin{GenesysTable}{Arcane Spell Components}{arcane-components}{ =l +l +X}
Component               & Rarity    & Description \\
Roc's Talon             & 6         & When casting an Arcana spell, adding the first Range effect
                                                    added to the spell does not increase the spell's difficulty.\\
Anakore King's Spleen   & 9         & When casting a spell, the caster may count any additional
                                                    \success beyond those needed to hit as \advantage\advantage\advantage
                                                    required to activate any added effect.\\
Braxat's Liver          & 6         &  When the user casts a spell, adding the Additional Target
                                                    effect does not increase its difficulty. In addition,
                                                    attack spells cast by the user increase their base
                                                    damage by three.\\
Elemental's Heart       & 3         & The cost to activate the elemental's quality is reducede to \advantage.
\end{GenesysTable}
\begin{GenesysTable}{Primal Spell Components}{primal-components}{ =l +l +l +X}
Component           & Cost      & Rarity    & Description \\
Raw Obsidian        & 800       & 7         & When casting an Conjure spell to summon an elemental,
                                                adding the Summon Ally effect does not increas its
                                                difficulty. In addition, the creature remains
                                                summoned untill the end of the encounter without
                                                your character having to use a concentrate manouvre.\\
Brain Seed          & 1000      & 6         & When the user casts a Conjure spell, adding Additional
                                                Summon effects do not increase its difficulty.\\
Need Root           &  900      & 5         & When the user casts an Augment spell, adding Additional
                                                Target effects do not increase its difficulty.\\
\end{GenesysTable}
\end{table*}

\subsubsection{Talisman of the Gaj} \label{itmmgc:talisman-of-the-gaj}
These highly sought after talismans are crafted by master craftsmen
from the mandibles of a Gaj. these items are heavily restricted because
of their power to cause havock.

If the wearer of the Talisman of the Gaj is a psionic, she gains
the \iqtyref{cortosis} quality, which means that any armor he wears
gains the \iqtyref{cortosis} quality and his Brawl attacks gain the
\iqtyref{cortosis} quality. Furthermore, his Brawl attacks gain
\iqtyref{pierce} X, where X is his current Psionic rank.

\end{multicols}
\FloatBarrier
