/FloatBarrier
\subsection{Magical and Psionics Gear}
\begin{multicols}{2}

\begin{table}[H]
\begin{GenesysTable}{Magic and Psionic Gear}{magic-and-psionic-gear}{ =X +l +l +l}
Name                                         & Encumbrance & Price & Rarity \\
\nameref{itmmgc:healingfruit}                & 0           & 30    & 5 \\
\nameref{itmmgc:spellcomponentpouch}         & 1           & 50    & (I) 3 \\
\end{GenesysTable}
\end{table}

\subsubsection{Healing Fruit} \label{itmmgc:healingfruit}
Healing Fruits are fruit grown from magical infused fruit trees.
While using any fruit is possible, often pears are most common.
While they are great for infusing a living body with healing,
their effects do diminish quite fast. The first healing fruit
eaten on a day grants 5 healing, the second eaten the same day
4 healing and so forth.

\subsubsection{Spell Component Pouch}
\label{itmmgc:spellcomponentpouch}
A spell component pouch contains the spell components neccesary to cast Arcane
Spells. A spell component pouch can contain one or more components, which must
be bought seperately or be sought out. An spell component is empty when a \despair
is used to 'damage' the spell component using a magic skill check, in which case
the spell component cannot be used for that scene.

Only one spell component can be used per casting, and doing so is included in the
casting action.

See~\tableref{spell-components} for a list for Spell Components.

\fxnote{\currentname: Create more and better named spell components}

\paragraph{Arcane Spell Components}

In addition to the usual exhausted rule for Spell Components, Arcane Spell
Components can be damaged. When a specific component is damaged 3 times, it
is destroyed. In which case it must be resought or bought again.

\begin{table*}[!htb]
\begin{GenesysTable}{Spell Components}{spell-components}{ =l +l +l +X}
Component   & Cost      & Rarity    & Description \\
S2          & (I) 400   & 6         & When casting a spell, adding the first Range effect added
                                            to the spell does not increase the spell's difficulty.\\
S3          & (I) 1000  & 9         & When casting a spell, the caster may count any additional
                                            \success beyond those needed to hit as \advantage\advantage\advantage
                                            required to activate any added effect.\\
S4          & 1000cp    & 6         &  When the user casts a spell, adding the Additional Target
                                        effect does not increase its difficulty. In addition,
                                        attack spells cast by the user increase their base
                                        damage by three.\\
%S4          &          & Rarity     & Attack spells cast by the user increase their base damage by 4.\\
\end{GenesysTable}
\end{table*}

\paragraph{Primal Spell Components}
\label{itmmgc:primal_implements}

\begin{table*}[!htb]
\begin{GenesysTable}{Primal Implements}{primal_implements}{ =l +l +l +X}
Component    & Cost      & Rarity    & Description \\
Raw Obsidian & (I) 800   & 7         & When casting an conjure spell to summon an elemental,
                                            adding the Summon Ally effect does not increas its
                                            difficulty. In addition, the creature remains
                                            summoned untill the end of the encounter without
                                            your character having to use a concentrate manouvre.\\
\end{GenesysTable}
\end{table*}


\end{multicols}
\FloatBarrier
