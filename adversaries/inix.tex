\subsection{Inix}

\Creature[name=Inix, type=Rival, image=images/inix.png, brawn=4, agility=1, intellect=1, willpower=2, presence=1, soak=4, wounds=12, meleedef=0, rangeddef=0, magicdef=0, psionicdef=0]{
    \textit{
        This large lizard has a long, curling tail and a broad mouth
        that looks quite capable of swallowing a halfling in a single
        gulp.
    }\\
    \\
    Inixes make good mounts because of the amount of
    weight they can carry for their size, which is two times
    their normal capacity.\\
    \\
    In combat inixes are fierce enemies. They usually
    attack with their tail first, taking advantage of its
    increased reach, then move in and try to bite, hoping to be
    able to grapple or swallow whole their victim.
} {
    % skill
    Athletics 3, Brawl 2, Resilience 1, Survival 1
}{
    % talents
    None
}{
    \\ % ability
    \textbf{- Large:} Silhouette 3\\
    \textbf{- Trained Mount 2:} ( Add \boost\boost to a rider's Riding check while mounted\\
    \textbf{- Sand Walker:} ( Remove 1 \setback from any checks made to traverse sandy or desert terrain\\
    \textbf{- Beast of Burden 10:} ( Add 10 to encumbrance threshold\\
    \textbf{- Grapple: } A Kank may spend \advantage\advantage after an attack
         and enemies must spend two maneuvers to disengage from an Inix.\\
}{
    \\ % equipment
    \textbf{- Trample:} Brawl; Damage: 8; Critical: 3; Range [Engaged], \iqtyref{inaccurate} 1, \iqtyref{knockdown}\\
    \textbf{- Tail Slap:} Brawl; Damage: 3; Critical: 4; Range [Engaged], \iqtyref{knockdown}, \iqtyref{disorient} 2\\
    \textbf{- Bite:} Brawl; Damage: 4; Critical: 5; Range [Engaged], Grapple\\
}
