\chapter{Environment}\label{chap:environment}

\textit{"The Tablelands are arid, hot, and barren. Even on windless days,
the sky is filled with a yellow-green haze of floating silt. The crimson sun
blazes with merciless intensity, and the breeze feels like the hot breath of
the Dragon itself."} - The Wanderer’s Journal\\

\section{Dessert Primer}
Athas is a desert world, but that doesn’t mean the planet is uniformly covered
with sand or barren wastes. Deserts come in many forms. Some are habitable,
some are brutal killing grounds, and some are wastelands that seem empty but
are full of hidden life. Knowing the types of deserts one might encounter
while traveling across Athas is a vital survival skill - one that might mean
the difference between a successful journey and a hard death in the wilds.

\begin{multicols}{2}

\subsection{Boulder Fields}

Boulder fields consist of broken, jagged rock. Some are old lava flows long
since cooled, and others are valleys choked with rockslides or slopes of scree.
They usually lie near mountains, and most are no larger than a few miles across.
Boulder fields are for- midable obstacles since they lack water, vegetation,
and shade, and if travelers do not have sturdy boots or sandals, the sharp
rocks can cut their feet to ribbons. Deep gulches and crevices crisscross
boulder fields, offering plenty of hiding places.

\subsection{Dust Sinks}

Windblown dust, ash, and silt accumulate in depressions to form dust sinks or
silt basins. The largest known example is the Sea of Silt, but smaller sinks
exist in almost any low-lying terrain. Even a light wind stirs the dust into
billowing clouds. On calm days, a dust sink appears to be a smooth plain of
pale gray or dun powder. Appearances are deceptive. The dust is too light to
support a traveler's weight, but it is thick enough to suffocate anyone who
falls in. Sometimes, the ground beneath the powder is uneven, concealing a
dangerous drop. One misstep, and a traveler can disappear beneath the dust.
Large bodies of silt often extend like the rivers of old into more solid
terrain, following narrow channels called estuaries. Many estuaries of silt
are shallow enough for human-sized travelers to wade with care. Very tall
creatures such as giants can navigate correspondingly deeper silt; a giant
can wade through silt 10 feet deep without difficulty. Many large sinks and
estuaries are sprinkled with islands of high ground, isolated from the
"mainland" by stretches of dust of varying depths. Some of these islands are
rocky protrusions just large enough to accommodate a giant or two, and others
can support an entire village. Miles of silt have sheltered many islands over
the years from the touch of defiling magic, and those islands remain
surprisingly verdant.

\subsection{Mountains}

Low ranges such as the Mekillot Mountains, the Stormclaw Mountains, and the
Black Spine Mountains dot the Tyr Region. They are daunting obstacles. Their
bare, rocky peaks - sometimes as tall as 2,000 meter — offer little water or
shelter to make the climb worthwhile. After a daytime temperature of well over
40 degrees Celcius, temperatures at night can plunge near the freezing point.
Most of the exposed rock crumbles under the twin hammers of heat and cold, so
great slopes of broken rock and frequent rockslides make for arduous travel.
Mountain vales, on the other hand, often are watered and filled with heavy
scrub, cacti, or sparse forest. Little of the land is suitable for
cultivation, but savages and monsters such as half-giants, gith, and kirres
make their homes in vales. Large networks of caverns lie under most of the low
mountain ranges, home to all sorts of strange creatures that prefer to hide
from the sun. A truly awesome mountain range marks the western border of the
Tyr Region - the Ringing Mountains, whose highest peaks reach 6,000 meter or
more. Some of these peaks have thin but permanent snowcaps.

\subsection{Mudflats}

Little open water remains on the surface of Athas; most is buried underground.
In a few places, water seeps upward, saturating the land to create mudflats.
Most common near or in dust sinks (especially the shallows of the Sea of Silt),
mudflats hide beneath the churning dust, revealed only when the winds clear an
area and expose the soupy mess to the air. Uncovered mudflats usually dry out
in short order, leaving behind hard, cracked clay that might or might not be
solid enough to support a traveler's weight. A few mudflats manage to survive,
sometimes through cultivation and sometimes by happenstance. These areas are
lush with vegetation, including desert grasses, thorny bushes, and small trees.
Where mudflats stand in silt basins, low islands of dense vegetation rise above
the dust. These mudflats are rarely large; most measure only a few hundred feet
across. Tangled underbrush and mucky ground make traveling through these areas
difficult but not impossible. In general, mudflats offer little to travelers;
there isn't much standing water, and dangerous predators hunt creatures that
subsist on the greenery.

\subsection{Rocky Badlands}

Most hilly regions on Athas are rocky badlands - highly eroded mazes of
sharp-edged ridges, winding canyons, and thorn-choked ravines. Daunting
escarpments force travelers into meandering courses along the ravine
floors, which often end in blind canyons or loop back on themselves.
Badlands can be barren, waterless wastes, but many are filled with thorny brush
that can completely clog the ravine floors. Rocky badlands are difficult to cross,
no matter which way a traveler means to go. Sticking to a canyon's floor is easy
enough, but a canyon rarely leads in the direction one desires, and the thick,
prickly brush makes for very hard going. Climbing up the walls to crest a badland
ridge usually involves a dangerous scramble of several hundred feet, and travel
along the top of a knife-edged ridge is equally challenging.

\subsection{Salt Flats}

Great flat plains encrusted with salt that is white, brown, or black, salt flats
can extend for miles. Some are dotted with briny marshland, but most are barren
and lifeless. Any water is usually too brackish to drink and might be poisonous.
Salt flats offer no shelter, and the temperatures reach more brutal extremes
than anywhere else on Athas. Sun sickness can kill an unprotected traveler
caught in a salt flat. If the salt flats have one asset, it's that no creatures
linger in them for long. A prepared traveler can cross a flat without risking
an encounter with a wild beast or roving band.

\subsection{Salt Marshes}

Salt marshes and shallow, ephemeral lakes can form in and near salt flats, dust
sinks, and sandy wastes. Most are only a mile or two across, but a few - such as
the Salt Meres or the Maze of Draj - extend for as much as hundreds of miles.
The water, too salty or alkaline to sustain life, is undrinkable. Many salt
marshes dry out completely in the months of High Sun, and some remain dry
year-round if the following Lowsun comes and goes without rain. A salt marsh
contains low grasses, reeds, or brush. Ankle-deep channels of briny water
encrusted with caked salt wind through the marsh, sometimes opening out into
large, shallow lakes. Here and there, tough stands of scrub or the occasional
tree stand above the grasses. Few creatures can digest the tough vegetation,
     but the marshes buzz with tiny insects that can drive a traveler half mad.

\subsection{Sandy Wastes}

Vast stretches of yellow sand, sandy wastes are the most identifiable deserts
of Athas. Some wastes are plains where the air is still and no winds disturb
the trackless land. In other wastes, the landscape takes on a rumpled appearance
as winds pile up sand to form great dunes. The topography of such wastes
changes endlessly; old dunes slowly erode under the wind, and new ones form
when deadly sandstorms whip up with little warning. Travelers caught in a storm
hear the wind howl in a deafening scream while stinging sand bites their skin.
The worst storms can scour flesh from bones. In the flat areas of Athas, sandy
wastes do not hinder travel. Oases, wells, and stands of tough scrub can
sustain desert-dwelling creatures and people indefinitely. Flat sand is easy
for travelers, although a lack of landmarks increases the risk of becoming
lost. In areas that have dunes, travel is more challenging. Mekillot dunes,
named for their passing resemblance to the huge drakes, can be hundreds of feet
tall, but most dunes rise no higher than 30 meter. In wastes where the winds
shift or collide, star dunes might form. The ridges of these mounds extend away
from the main mass, forming arms that spread out like tentacles in all directions.

\subsection{Scrub Plains}

Scrub plains are savanna, prairie, or chaparral with just enough water to
support extensive vegetation. Tough, dry grass punctuated by creosote bushes
and tumbleweed dominates the ground. One can even find a few small trees
scattered across the landscape. By Athasian standards, scrub plains are almost
lush, supporting a high concentration of wildlife. Excessive grazing and the
use of defiling magic have reduced some scrub plains in the Tyr Region to ruin.
Only a few such areas survive in the wild lands between the city-states,
protected by primal guardians who use ancient magic to destroy intruders and
safeguard their homes. However, beyond the Ringing Mountains stretch vast
scrub plains such as the Crimson Savanna.

\subsection{Stony Barrens}

Stony barrens dominate the Tablelands. Most barrens are bedrock shelves exposed
by windstorms. These weathered plains are covered with rocks that range in size
from pebbles and gritty dust to huge piles of standing boulders. In places,
the bare rock gives way to hard-packed red earth, and yellow sand collects in
crevices, forming dunes or drifts. Huge mesas and pointed buttes dot the plains,
a testimony to the erosive power of the elements. Cacti proliferate in stony
barrens. Hundreds of species grow throughout, appearing in all shapes and
sizes, from small, thorny buttons to towering saguaros. Some cacti are edible,
making suitable fare for travelers low on supplies. Others are stealthy
predators that can kill careless travelers; in the Athasian wilderness, one
can never be certain who is the hunter and who is the hunted.

\end{multicols}

\section{Environmental Effects}

\subsection{Concealment}
Concealment is a situation that occurs when a character is harder to spot because
of environmental effects such as darkness, fog or a sand storm. Concealment imposes
penalties on attacks and sight-based skill checks. Conversely it can provide bonuses
for other skill checks, such as Stealth.

As a general guide the following guide can be used to dermine the number of \setback
dice to add against targets with concealment, or \boost dice to add when engaging in
Stealth based checks. As a guide, using Melee skills in darkness of fog should add 1/2
the number of \setback dice indicated.

\begin{table*}[!htb]
\begin{GenesysTable}{Concealment}{concealment}{ =l +X}
Dice Added  & Examples\\
+1          & Mist, shadow, waist-high grass.\\
+2          & Fog, the darkness of early morning or late evening, thick, shoulder-high grass.\\
+3          & Sand storm, Heavy fog, thick and chocking smoke, the darkness of night, dense, head-high underbrush and thick grass.\\
\end{GenesysTable}
\end{table*}

\subsection{Cover}
Most cover adds a \setback against Ranged attacks. Being especially well covered
would add \setback\setback, such as shooting at a target hiding in a trench.

\subsection{Dehydration}
\subsection{Difficult and Impassable Terrain}
Difficult terrain is a catch-all description of terrain that is hard to move through
or over. It can include tight passageways, shifting sand or loose rubble. Essentially,
it's terrain that characters move through with difficulty. Characters entering or
moving through difficult terrain must perform twice as many maneuvers to move the
same distance they would in normal terrain.

\subsubsection{Quick Sand}

\textit{TODO: Add content}

\subsection{Fire, Acid, and Corrosive Atmosphere}
\textit{TODO: Add content}
\subsection{Falling}
\textit{TODO: Add content}
\subsection{Sandstorms}
\textit{TODO: Add content}
\subsection{Survival Day}
\textit{TODO: Add content}
\subsection{Sun Sickness}
\textit{TODO: Add content}

%SandStorms
%
%Blowing sand can quickly turn into a nasty sandstorm. There are two types of sandstorms - mild and driving.
%
%Mild sandstorms last 1d20 rounds. On a roll of 20, a mild sandstorm becomes a driving sandstorm after 10 rounds. In a mild sandstorm, visibility is reduced (see DS Rules Book p. 84). In addition, movement must be reduced to half speed or the party risks getting lost. If the party does not reduce its rate of movement, the lead character (or animal driver) must make a Wisdom check every round the storm lasts to stay on course. Don't tell the players that this is what the roll is for because, if they fail, they should not know that they have gone the wrong way. How badly a party fails its Wisdom check determines how much time is added to their trek. In mild sandstorms, the difference translates into hours. For example, Azhul the Hasty leads the way through a mild sandstorm. His Wisdom is 10, but he rolls a 14. The party travels an extra four hours (14- 10 = 4).
%
%Driving sandstorms last 1d10 rounds. In a driving sandstorm, visibility is reduced and movement must be reduced to one-quarter speed to keep from getting lost. Wisdom checks are needed every round as described above, but these checks are made at +5. How badly the party fails determines how much time is added to travel in the form of days. For example, if Azhul, with his Wisdom of 10, rolls a 7, the party adds two days to their trek (7 + 5 - 10 = 2).
%
%Source: Arcane Shadows, Part Three: E - The Wilderness
%
%Sandstorm effects
%
%Characters with the Weather Sense nonweapon proficiency should make four checks as the day progresses. Characters who succeed at least three times realize that a sandstorm is coming in time for the PCs to find shelter.
%
%Characters who succeed only two of the checks know that a sandstorm is imminent, but must endure the storm as best they can with minimum preparations. In this case, they can simply stay where they are or try to keep moving. Moving through a sand storm, even when prepared, is a tricky proposition. The lead character must reduce the party's movement to one-quarter speed or risk getting lost. A Wisdom check must be made with a -5 penalty. Success means the party stays on course. Failure indicates they get lost. How badly the check fails determines the number of days that must be added to the PC's trek. For example, if a character with a Wisdom of 12 leads the way through the sandstorm and rolls an 8, the party adds one day to its trek 8 + 5 - 12 = 1). Characters with the Direction Sense proficiency or Know Direction psionic talent suffer no penalties to their checks.
%
%Characters who succeed at only one check or less are caught by surprise when the sandstorm hits. In addition to possibly getting lost, the party also suffers the effects listed on the table below (DM rolls 1d100).
%
%1d100 Roll Effect
%
%01-40 Party manages to ride out the storm. Apart from exhaustion and possibly getting lost, no other effects.
%
%41-65 Party loses two pieces of equipment.
%
%66-80 Party loses two pieces of equipment plus 80% of its water.
%
%81-95 Every character and riding animal takes 2d6 damage and items are lost on a roll of 1-3 on 1d6.
%
%96-00 Every character and riding animal takes 4d6 damage and items are lost on a roll of 1-3 on 1d4. Every creature has a 4% chance of being buried alive.
%
%Source: Dragon's Crown, p.42
%
%Wind and Sand
%
%Dehydration is not the only enemy for those journeying through the desert. High winds can lift sand and dust into a choking, blinding storm that can scour individuals as well as property. Characters trapped in such a storm without protection suffer 1d2 points of damage per round. In addition, they must make a saving throw vs. wands; those who fail are blinded (per the spell) for 1d6 turns. A tent or rock outcropping offers sufficient protection from the storm; so does lying prone with a cloth across the eyes, nose, and mouth. Further, the protection from normal missiles spell and similar magics can protect the individual unless the storm is magical in origin.
%
%In addition to inflicting the damage noted above, desert storms can bury characters alive, eventually causing them to suffocate. So can certain spells that trigger sandslides or move dunes. (See Chapter 8 for details on spells.) Characters who are buried alive by a desert storm can dig themselves out in 1d3 rounds. Those buried by an avalanche of sand — whether natural or caused by a spell — can dig free in 1d6 rounds unless otherwise noted in the spell description.
%
%Crawling out of a sandy grave is no simple task. For each round spent digging toward the surface, a character must make a Strength check as well as a Constitution check. A successful Strength check reduces the time required for escape by one round; failure has no effect. In contrast, a failed Constitution check results in the loss of 1d4 points of Constitution, while a successful Constitution check neither helps nor hinders the character. An individual reduced to 0 Constitution cannot move. If no help is forthcoming, the paralyzed character will suffocate in 1d10 rounds.
%
%A number of variables can delay or retard suffocation, however, including spells and magical items which reduce or eliminate the need to breathe. The endurance proficiency enables a character to make a Constitution check every other round instead of every round, but it does not affect the required Strength checks.
%
%Assuming they know where to dig, other characters can rescue an individual who has been buried alive. For every round in which they dig downward, wouldbe rescuers reduce the number of rounds required for escape by one. Excavating time is the same no matter how many characters dig. Rescuers can dig out an individual who has reached 0 Constitution, and is unable to move.
%
%Constitution lost while a character is buried alive is regained at 1 point per turn. Hit points are unaffected by Constitution lost in this fashion. Constitution may never be regained to a level higher than a character’s usual maximum.
