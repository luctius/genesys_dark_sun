\chapter{Rules Overview}\label{chap:rules_overview}

\begin{centering}
    \textit{This is a very rough summary of the Core Rules. Refer to the Genesys Core Rulebook for the full Rules.}
\end{centering}

\begin{multicols}{2}

\subsection{Turns}
\subsubsection{Manouvers}
Maneuvers are activities that aren’t complex enough to
warrant a skill check, but still involve time and effort
on the part of a character. Characters are allowed one
maneuver during their turn, and certain circumstances
may allow them a second maneuver as well, such as spending two \advantage.
The following are some examples of maneuvers:
\begin{description}
    \item Aiming a weapon (\boost per aim maneuver).
    \item Assisting an ally (Add \boost to that characters specific check per ally helping).
    \item Moving one range band closer or farther away from an enemy.
    \item Opening a door.
    \item Diving behind cover.
    \item Flipping over a table.
    \item Standing up or dropping prone.
    \item Mount or dismount.
    \item Guarded Stance. (gain \setback and Melee Defense +1 until the end of their next turn)
\end{description}

\subsubsection{Actions}
Actions are important activities that are vital to a character's accomplishment
of a goal. Each character may normally only perform one action during their
turn, likely the most important activity they undertake during their turn.
Actions almost always involve performing a skill check, although certain
character abilities may require using an action to activate them.

The following are some examples of actions:
\begin{description}
    \item Unlocking a locked door.
    \item Shooting a bow
    \item Punching or grappling an opponent.
    \item Instructing allies with a series of orders.
    \item Performing first aid on an ally.
    \item Sneaking up on a vigilant foe.
    \item Climbing a cliff.
\end{description}
Out of all of these options, the most common during combat are those that
involve attacking an opponent. Attacking an opponent requires a combat
skill check, sometimes referred to in shorthand as a combat check or
simply an attack.

\subsection{Performing a Combat Check}
\begin{table}[H]
\begin{GenesysTable}{Ranged Attack Difficulties}{attack-ranged-difficulties}{ =l +X}
Range Band  & Difficulty\\
Engaged     & \textbf{Easy} (\difficulty) plus modifiers depending of the weapon used.\\
Short       & \textbf{Easy} (\difficulty)\\
Medium      & \textbf{Average} (\difficulty\difficulty)\\
Long        & \textbf{Hard }(\difficulty\difficulty\difficulty)\\
Extreme     & \textbf{Daunting }(\difficulty\difficulty\difficulty\difficulty)\\
\end{GenesysTable}
\end{table}

\subsubsection{Spending Advantage and Triumph in Combat}
\begin{table*}[!htb]
\begin{GenesysTable}{Spending Advantage and Triumph in Combat}{attack-spending-advantage}{ =l +X}
Cost        & Result Option\\
\advantage or \triumph  & Recover 1 strain.\newline
                          Add \boost to the next allied character's check\newline
                          Notice a single important point in the ongoing conflict, such as a weakspot in the armour.\newline
                          Inflict a Critical Injury with a successful attack that deals damage poast soak (\advantage cost may vary).\newline
                          Activate an item quality (\advantage cost may vary).\\
\advantage\advantage or \triumph  & Perform an immediate free maneuver that does not exceed the limit of two maneuvers per turn.\newline
                                    Add \setback to the targeted character's next check.\newline
                                    Add \boost to any allied character's next check, including that of the active character.\\
\advantage\advantage\advantage or \triumph  & Negate the targeted enemy's defense (such as the defense gained from cover, equipment, or performing the guarded stance
                                              maneuver) until the end of the current round.\newline
                                              Ignore penalizing environmental effects such as inclement weather, zero gravity, or similar circumstances until the end of the
                                              active character’s next turn.\newline
                                              When dealing damage to a target, have the attack disable the opponent or one piece of gear rather than dealing wounds or strain.\newline
                                              This could include hobbling them temporarily with a shot to the legThis should be agreed upon by the
                                              player and the GM, and the effects are up to the GM. The effects should be temporary and not too excessive.\newline
                                              Gain +1 melee or ranged defense until the end of the active character's next turn.\newline
                                              Force the target to drop a melee or ranged weapon they are wielding.\\
\triumph  &         Upgrade the difficulty of the targeted character’s next check.\newline
                    Upgrade the ability of any allied character’s next check, including that of the current active character.\newline
                    Do something vital, such as shooting the controls to the nearby blast doors to seal them shut.\newline
                    On an Initiative check, perform an immediate free maneuver before combat begins.\\
\triumph\triumph  & When dealing damage to a target, have the attack destroy a
                    piece of equipment the target is using, such as blowing up
                    their assault rifle or slicing their sword in half. \\
\end{GenesysTable}
\end{table*}

\subsubsection{Spending Setback and Despair in Combat}
\begin{table*}[!htb]
\begin{GenesysTable}{Spending Setback and Despair in Combat}{attack-spending-setback}{ =l +X}
Cost        & Result Option\\
\setback or \despair  & The active character suffers 1 strain.\newline
                          The active character loses the benefits of a prior maneuver (such as from taking cover or assuming a guarded stance) until they perform the maneuver again.\\
\setback\setback or \despair  & An opponent may immediately perform one free maneuver as an incidental in response to the active character's check.\newline
                                    Add \boost to the targeted character's next check.\newline
                                    The active character or an allied character suffers \boost on their next action.\\
\setback\setback\setback or \despair  & The active character falls prone.\newline
                                              The active character grants the enemy a significant advantage in the ongoing
                                              encounter, such as accidentally blasting the controls to a bridge the active
                                              character was planning to use for their escape.\\
\despair  & Upgrade the difficulty of an allied character's next check or the next check of the current active character.\newline
            The tool, Brawl, or Melee weapon the character is using becomes damaged.\\
\despair\despair  & The character's weapon immediately breaks if it has the \iqtyref{fragile} Quality.\\
                    
\end{GenesysTable}
\end{table*}

\end{multicols}

\FloatBarrier

\section{Injury}
\subsection{Critical hits}

\begin{table*}[!htb]
\begin{GenesysTable}{Critical Hits}{critical-hits}{ =l +l +X}
D100    & Severity              & Result\\
01-05   & \textbf{Easy} (\difficulty)    & Minor Nick: The target suffers 1 strain.\\
06-10   & \textbf{Easy} (\difficulty)    & Slowed Down: The target can only act during the last allied Initiative slot on their next turn.\\
11-15   & \textbf{Easy} (\difficulty)    & Sudden Jolt: The target drops whatever is in hand.\\
16-20   & \textbf{Easy} (\difficulty)    & Distracted: The target cannot perform a free maneuver during their next turn.\\
21-25   & \textbf{Easy} (\difficulty)    & Off-Balance: Add  to the target’s next skill check.\\
26-30   & \textbf{Easy} (\difficulty)    & Discouraging Wound: Move one player pool Story Point to the Game Master pool (reverse if NPC).\\
31-35   & \textbf{Easy} (\difficulty)    & Stunned: The target is staggered until the end of their next turn.\\
31-40   & \textbf{Easy} (\difficulty)    & Stinger: Increase the difficulty of the target’s next check by one.\\
41-45   & \textbf{Average} (\difficulty\difficulty)    & Bowled Over: The target is knocked prone and suffers 1 strain.\\
46-50   & \textbf{Average} (\difficulty\difficulty)    & Head Ringer: The target increases the difficulty of all Intellect and Cunning checks by one until this Critical Injury is healed.\\
51-55   & \textbf{Average} (\difficulty\difficulty)    & Fearsome Wound: The target increases the difficulty of all Presence and Willpower checks by one until this ritical Injury is healed.\\
51-60   & \textbf{Average} (\difficulty\difficulty)    & Agonizing Wound: The target increases the difficulty of all Brawn and Agility checks by one until this Critical Injury is healed.\\
61-65   & \textbf{Average} (\difficulty\difficulty)    & Slightly Dazed: The target is disoriented until this Critical Injury is healed.\\
61-70   & \textbf{Average} (\difficulty\difficulty)    & Scattered Senses: The target removes all  from skill checks until this Critical Injury is healed.\\
71-75   & \textbf{Average} (\difficulty\difficulty)    & Hamstrung: The target loses their free maneuver until this Critical Injury is healed.\\
71-80   & \textbf{Average} (\difficulty\difficulty)    & Overpowered: The target leaves themself open, and the attacker may immediately attempt another attack against them as an incidental, using the exact same pool as the original attack.\\
81-85   & \textbf{Average} (\difficulty\difficulty)    & Winded: The target cannot voluntarily suffer strain to activate any abilities or gain additional maneuvers until this Critical Injury is healed.\\
81-90   & \textbf{Average} (\difficulty\difficulty)    & Compromised: Increase difficulty of all skill checks by one until this Critical Injury is healed.\\
91-95   & \textbf{Hard} (\difficulty\difficulty\difficulty)    & At the Brink: The target suffers 2 strain each time they perform an action until this Critical Injury is healed.\\
91-100  & \textbf{Hard} (\difficulty\difficulty\difficulty)    & Crippled: One of the target’s limbs (selected by the GM) is impaired until this Critical Injury is healed. Increase difficulty of all checks that require use of that limb by one.\\
101-105 & \textbf{Hard} (\difficulty\difficulty\difficulty)    & Maimed: One of the target’s limbs (selected by the GM) is permanently lost. Unless the target has a cybernetic or prosthetic replacement, the target cannot perform actions that would require the use of that limb. All other actions gain  until this Critical Injury is healed.\\
106-110 & \textbf{Hard} (\difficulty\difficulty\difficulty)    & Horrific Injury: Roll 1d10 to determine which of the target’s characteristics is affected: 1–3 for Brawn, 4–6 for Agility, 7 for Intellect, 8 for Cunning, 9 for Presence, 10 for Willpower. Until this Critical Injury is healed, treat that characteristic as one point lower.\\
111-115 & \textbf{Hard} (\difficulty\difficulty\difficulty)    & Temporarily Disabled: The target is immobilized until this Critical Injury is healed.\\
116-120 & \textbf{Hard} (\difficulty\difficulty\difficulty)    & Blinded: The target can no longer see. Upgrade the difficulty of all checks twice, and upgrade the difficulty of Perception and Vigilance checks three times, until this Critical Injury is healed.\\
121-125 & \textbf{Hard} (\difficulty\difficulty\difficulty)    & Knocked Senseless: The target is staggered until this Critical Injury is healed.\\
126-130 & \textbf{Daunting} (\difficulty\difficulty\difficulty\difficulty)    & Gruesome Injury: Roll 1d10 to determine which of the target’s characteristics is affected: 1–3 for Brawn, 4–6 for Agility, 7 for Intellect, 8 for Cunning, 9 for Presence, 10 for Willpower. That characteristic is permanently reduced by one, to a minimum of 1.\\
131-140 & \textbf{Daunting} (\difficulty\difficulty\difficulty\difficulty)    & Bleeding Out: Until this Critical Injury is healed, every round, the target suffers 1 wound and 1 strain at the beginning of their turn. For every 5 wounds they suffer beyond their wound threshold, they suffer one additional Critical Injury. Roll on the chart, suffering the injury (if they suffer this result a second time due to this, roll again).\\
141-150 & \textbf{Daunting} (\difficulty\difficulty\difficulty\difficulty)    & The End Is Nigh: The target dies after the last Initiative slot during the next round unless this Critical Injury is healed.\\
151+    &                       & Dead: Complete, obliterated death.\\
\end{GenesysTable}
\end{table*}

\FloatBarrier

\section{Environmental Effects}

\begin{multicols}{2}

\subsection{Concealment}
Concealment is a situation that occurs when a character is harder to spot because
of environmental effects such as darkness, fog or a sand storm. Concealment imposes
penalties on attacks and sight-based skill checks. Conversely it can provide bonuses
for other skill checks, such as Stealth.

As a general guide the following guide can be used to dermine the number of \setback
dice to add against targets with concealment, or \boost dice to add when engaging in
Stealth based checks. As a guide, using Melee skills in darkness of fog should add 1/2
the number of \setback dice indicated.

\begin{table}[H]
\begin{GenesysTable}{Concealment}{concealment}{ =l +X}
Dice Added  & Examples\\
+1          & Mist, shadow, waist-high grass.\\
+2          & Dust Storm, Fog, the darkness of early morning or late evening, thick, shoulder-high grass.\\
+3          & Sand storm, Heavy fog, thick and chocking smoke, the darkness of night, dense, head-high underbrush and thick grass.\\
\end{GenesysTable}
\end{table}

\subsection{Cover}
Most cover adds a \setback against Ranged attacks. Being especially well covered
would add \setback\setback, such as shooting at a target hiding in a trench.

\subsection{Difficult and Impassable Terrain}
Difficult terrain is a catch-all description of terrain that is hard to move through
or over. It can include tight passageways, shifting sand or loose rubble. Essentially,
it's terrain that characters move through with difficulty. Characters entering or
moving through difficult terrain must perform twice as many maneuvers to move the
same distance they would in normal terrain.

\subsection{Falling}

On Athas, gravity has a habit of ruining someones day. When a character falls,
consult \tableref{falling-damage} and apply the damage. Soak will reduce the
damage, however any strain damage is not reduced.

A character can reduce the damage taken from falling by makeing an
\textbf{Average} (\difficulty\difficulty) Athletics or Coordination Check.
Each \success reduces to damage sufferec by one while each \advantage
reduces to strain suffered by one. A \triumph could, at the GM's
discretion, reduce the overall distance fallen by one range band
as the character grabs onto a hanhold or does something else to
slow their fall.

\begin{table}[H]
\begin{GenesysTable}{Falling Damage}{falling-damage}{ =l +X +l}
Range   & Damage                                            & Strain\\
Short   & 10                                                & 10\\
Medium  & 30                                                & 20\\
Long    & Incapacitated, Critical Injury at +50             & 30\\
Extreme & Incapacitated, Critical Injury at +75 (or Death)  & 40\\
\end{GenesysTable}
\end{table}

\subsection{Heat and Cold}
While most settlements have plenty of heat cover and the Athasians cloth themselves
with heat in mind, working directly in the sun during the heat of day is exhausting.
Doing anything physical in the middle of the day without cover adds a \setback to all checks.
Wearing heavy armour will increas this to \setback\setback.

\subsection{Sandstorms}

In addition to causing concealment, sandstorms make it hard to navigate. Sandstorms
add \setback\setback\setback\setback to any attempt to navigate.

\subsection{Survival Rating}

\textbf{Survival Rating} are an abstract mechanism to simulate the need to prepare for journies in Dark Sun.
They are a group mechanic, as the whole group is expected to share it's resources. Any time there is
a roll while outside a settlement, and a \despair or \threat\threat\threat is rolled, the DM can use
that to remove 1 from the \textbf{Survival Rating} and narrate that as appropriate.

\begin{table}[H]
\begin{GenesysTable}{Survival Rating}{survival-rating}{ =l +c +c +c}
Journey Length & Encumbrance & Cost   & Rating\\
Short          &     1       &  20cp  &   5\\
Medium         &     2       &  40cp  &  10\\
Long           &     4       & 100cp  &  25\\
Very Long      &     8       & 200cp  &  50\\
\end{GenesysTable}
\end{table}

The group can call for \textbf{Rationing}, which will mean that it will take \despair\despair to remove 1
\textbf{Survival Rating}. However doing so will prohibit the group from recouperating any strain after any
scene until \textbf{Rationing} is stopped.

If the \textbf{Survival Rating} of the group hits half or less, the group will automatically go into \textbf{Rationing}.
If the \textbf{Survival Rating} drops to zero, all non-magical healing is prohibited.

\end{multicols}
