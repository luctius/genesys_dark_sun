\begin{table*}[!htb]
\centering
\small\caption{Barrier Additional Effects}
\begin{GenesysTable}{l X}
Cost                    & Effect\\
\difficulty             & \textbf{Additional Target:} The spell affects one additional
                            target within range of the spell. In addition, after
                            casting the spell, you may spend \advantage to affect
                            one additional target within range of the spell (and
                            may trigger this multiple times, spending \advantage
                            each time).\\
\difficulty             & \textbf{Range:} Increase the range of the spell by one range band.
                            This may be added multiple times, increasing the range
                            by one range band each time.\\
\difficulty\difficulty  & \textbf{Add Defense:} Each affected target gains ranged and melee
                            defense equal to your ranks in Knowledge.\\
\difficulty\difficulty  & \textbf{Empowered:} The barrier reduces damage equal to the
                            number of uncanceled \success instead of the normal
                            effect.\\
\end{GenesysTable}
\label{table:magic_barrier}
\end{table*}

%TODO: rephrease text
%TODO: replace Knowledge
\subsubsection{Barrier}
\textbf{Skill:} Arcane\\
\textbf{Concentration:} Yes\\
\textbf{Basic Difficulty:} \textbf{Easy:} (\difficulty)\\
Both arcane and divine spellcasters have the power to
create barriers of magical energy to protect themselves
and their allies. The character selects one target they are
engaged with (which can be themself), then makes an
Arcana or Divine skill check.  If the check is successful,
until the end of the character’s next turn, reduce the damage
of all hits the target suffers by one, and further reduce it
by one for every uncanceled \success\success beyond the first.
Before making an Barrier check, you may choose any number of
additional effects from ~\tref{table:magic_barrier}.

